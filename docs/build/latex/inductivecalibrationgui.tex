%% Generated by Sphinx.
\def\sphinxdocclass{report}
\documentclass[letterpaper,10pt,english]{sphinxmanual}
\ifdefined\pdfpxdimen
   \let\sphinxpxdimen\pdfpxdimen\else\newdimen\sphinxpxdimen
\fi \sphinxpxdimen=.75bp\relax
\ifdefined\pdfimageresolution
    \pdfimageresolution= \numexpr \dimexpr1in\relax/\sphinxpxdimen\relax
\fi
%% let collapsible pdf bookmarks panel have high depth per default
\PassOptionsToPackage{bookmarksdepth=5}{hyperref}

\PassOptionsToPackage{warn}{textcomp}
\usepackage[utf8]{inputenc}
\ifdefined\DeclareUnicodeCharacter
% support both utf8 and utf8x syntaxes
  \ifdefined\DeclareUnicodeCharacterAsOptional
    \def\sphinxDUC#1{\DeclareUnicodeCharacter{"#1}}
  \else
    \let\sphinxDUC\DeclareUnicodeCharacter
  \fi
  \sphinxDUC{00A0}{\nobreakspace}
  \sphinxDUC{2500}{\sphinxunichar{2500}}
  \sphinxDUC{2502}{\sphinxunichar{2502}}
  \sphinxDUC{2514}{\sphinxunichar{2514}}
  \sphinxDUC{251C}{\sphinxunichar{251C}}
  \sphinxDUC{2572}{\textbackslash}
\fi
\usepackage{cmap}
\usepackage[T1]{fontenc}
\usepackage{amsmath,amssymb,amstext}
\usepackage{babel}



\usepackage{tgtermes}
\usepackage{tgheros}
\renewcommand{\ttdefault}{txtt}



\usepackage[Bjarne]{fncychap}
\usepackage{sphinx}

\fvset{fontsize=auto}
\usepackage{geometry}


% Include hyperref last.
\usepackage{hyperref}
% Fix anchor placement for figures with captions.
\usepackage{hypcap}% it must be loaded after hyperref.
% Set up styles of URL: it should be placed after hyperref.
\urlstyle{same}

\addto\captionsenglish{\renewcommand{\contentsname}{Contents:}}

\usepackage{sphinxmessages}
\setcounter{tocdepth}{1}



\title{inductive calibration GUI}
\date{Dec 01, 2021}
\release{V1.0}
\author{Martijn Schouten}
\newcommand{\sphinxlogo}{\vbox{}}
\renewcommand{\releasename}{Release}
\makeindex
\begin{document}

\pagestyle{empty}
\sphinxmaketitle
\pagestyle{plain}
\sphinxtableofcontents
\pagestyle{normal}
\phantomsection\label{\detokenize{index::doc}}


\sphinxAtStartPar
This documentation documenents the code of a GUI for calibrating a Diabase H\sphinxhyphen{}Series 3D printer in x and y using and LDC1101EVM evaluation module. The easiest way to run a frozen binary which can be found in \sphinxhref{https://github.com/martijnschouten/inductive\_calibration\_GUI/releases}{releases}

\sphinxAtStartPar
The inductive calibraiton GUI consists of three classes. The mainwindow of the app contain the entire GUI. The diabase class implements the communication with the diabase 3D printer and the ldc1101evm class implements the communication with the LDC1101EVM evaluation module.


\chapter{App mainwindow class}
\label{\detokenize{index:app-mainwindow-class}}\phantomsection\label{\detokenize{index:module-app}}\index{module@\spxentry{module}!app@\spxentry{app}}\index{app@\spxentry{app}!module@\spxentry{module}}\phantomsection\label{\detokenize{index:module-0}}\index{module@\spxentry{module}!app@\spxentry{app}}\index{app@\spxentry{app}!module@\spxentry{module}}\index{MainWindow (class in app)@\spxentry{MainWindow}\spxextra{class in app}}

\begin{fulllineitems}
\phantomsection\label{\detokenize{index:app.MainWindow}}\pysiglinewithargsret{\sphinxbfcode{\sphinxupquote{class\DUrole{w}{  }}}\sphinxcode{\sphinxupquote{app.}}\sphinxbfcode{\sphinxupquote{MainWindow}}}{\emph{\DUrole{o}{*}\DUrole{n}{args}}, \emph{\DUrole{o}{**}\DUrole{n}{kwargs}}}{}
\sphinxAtStartPar
Bases: \sphinxcode{\sphinxupquote{PyQt5.QtWidgets.QMainWindow}}
\index{apply\_offsets() (app.MainWindow method)@\spxentry{apply\_offsets()}\spxextra{app.MainWindow method}}

\begin{fulllineitems}
\phantomsection\label{\detokenize{index:app.MainWindow.apply_offsets}}\pysiglinewithargsret{\sphinxbfcode{\sphinxupquote{apply\_offsets}}}{}{}
\sphinxAtStartPar
Function for handling the apply offset button being pressed. This will send the measured offsets to the printer.
:return: None
:rtype: None

\end{fulllineitems}

\index{ascend (app.MainWindow attribute)@\spxentry{ascend}\spxextra{app.MainWindow attribute}}

\begin{fulllineitems}
\phantomsection\label{\detokenize{index:app.MainWindow.ascend}}\pysigline{\sphinxbfcode{\sphinxupquote{ascend}}\sphinxbfcode{\sphinxupquote{\DUrole{w}{  }\DUrole{p}{=}\DUrole{w}{  }True}}}
\sphinxAtStartPar
If the tools should be calibrated in ascending (True) or descending (False) order.

\end{fulllineitems}

\index{ascend\_changed() (app.MainWindow method)@\spxentry{ascend\_changed()}\spxextra{app.MainWindow method}}

\begin{fulllineitems}
\phantomsection\label{\detokenize{index:app.MainWindow.ascend_changed}}\pysiglinewithargsret{\sphinxbfcode{\sphinxupquote{ascend\_changed}}}{}{}
\sphinxAtStartPar
Function for handling the ascend checkbox being pressed. This will update the ascend setting and deselect the descend checkbox.
:return: None
:rtype: None

\end{fulllineitems}

\index{calibrate() (app.MainWindow method)@\spxentry{calibrate()}\spxextra{app.MainWindow method}}

\begin{fulllineitems}
\phantomsection\label{\detokenize{index:app.MainWindow.calibrate}}\pysiglinewithargsret{\sphinxbfcode{\sphinxupquote{calibrate}}}{\emph{\DUrole{n}{cal\_x}}}{}
\sphinxAtStartPar
Function for performing a calibration in x or y. This will just record the LDC1101EVM sensor values until the stop button is clicked and store the result in the file specified in the filename textbox.
:param cal\_x: If True, calibrate in the x direction. If False, calibate in the y direction.
:return: False if unsucceful, True if succefull
:rtype: Boolean

\end{fulllineitems}

\index{calibrate\_x() (app.MainWindow method)@\spxentry{calibrate\_x()}\spxextra{app.MainWindow method}}

\begin{fulllineitems}
\phantomsection\label{\detokenize{index:app.MainWindow.calibrate_x}}\pysiglinewithargsret{\sphinxbfcode{\sphinxupquote{calibrate\_x}}}{}{}
\sphinxAtStartPar
Function for handling the calibrate x button being pressed. This will run the calibration procedure and find the x offsets.
:return: None
:rtype: None

\end{fulllineitems}

\index{calibrate\_y() (app.MainWindow method)@\spxentry{calibrate\_y()}\spxextra{app.MainWindow method}}

\begin{fulllineitems}
\phantomsection\label{\detokenize{index:app.MainWindow.calibrate_y}}\pysiglinewithargsret{\sphinxbfcode{\sphinxupquote{calibrate\_y}}}{}{}
\sphinxAtStartPar
Function for handling the calibrate y button being pressed. This will run the calibration procedure and find the y offsets.
:return: None
:rtype: None

\end{fulllineitems}

\index{clear\_figure() (app.MainWindow method)@\spxentry{clear\_figure()}\spxextra{app.MainWindow method}}

\begin{fulllineitems}
\phantomsection\label{\detokenize{index:app.MainWindow.clear_figure}}\pysiglinewithargsret{\sphinxbfcode{\sphinxupquote{clear\_figure}}}{}{}
\sphinxAtStartPar
Function for handling the clear figure button being pressed. This will clear the graph in the GUI and reinitialise it.
\begin{quote}\begin{description}
\item[{Returns}] \leavevmode
\sphinxAtStartPar
None

\item[{Return type}] \leavevmode
\sphinxAtStartPar
None

\end{description}\end{quote}

\end{fulllineitems}

\index{closeEvent() (app.MainWindow method)@\spxentry{closeEvent()}\spxextra{app.MainWindow method}}

\begin{fulllineitems}
\phantomsection\label{\detokenize{index:app.MainWindow.closeEvent}}\pysiglinewithargsret{\sphinxbfcode{\sphinxupquote{closeEvent}}}{\emph{\DUrole{n}{event}}}{}
\sphinxAtStartPar
Function for handling the window being closed. This makes sure the settings are saved when the window is closed.
:return: None
:rtype: None

\end{fulllineitems}

\index{connect() (app.MainWindow method)@\spxentry{connect()}\spxextra{app.MainWindow method}}

\begin{fulllineitems}
\phantomsection\label{\detokenize{index:app.MainWindow.connect}}\pysiglinewithargsret{\sphinxbfcode{\sphinxupquote{connect}}}{}{}
\sphinxAtStartPar
Function for handling the connect button being pressed. This will attempted to connect to the selected COM ports.
\begin{quote}\begin{description}
\item[{Returns}] \leavevmode
\sphinxAtStartPar
True if succesfull, False if unsuccesfull

\item[{Return type}] \leavevmode
\sphinxAtStartPar
Boolean

\end{description}\end{quote}

\end{fulllineitems}

\index{connected (app.MainWindow attribute)@\spxentry{connected}\spxextra{app.MainWindow attribute}}

\begin{fulllineitems}
\phantomsection\label{\detokenize{index:app.MainWindow.connected}}\pysigline{\sphinxbfcode{\sphinxupquote{connected}}\sphinxbfcode{\sphinxupquote{\DUrole{w}{  }\DUrole{p}{=}\DUrole{w}{  }False}}}
\sphinxAtStartPar
If a connection to the LDC1101EVM and diabase has already been made

\end{fulllineitems}

\index{descend\_changed() (app.MainWindow method)@\spxentry{descend\_changed()}\spxextra{app.MainWindow method}}

\begin{fulllineitems}
\phantomsection\label{\detokenize{index:app.MainWindow.descend_changed}}\pysiglinewithargsret{\sphinxbfcode{\sphinxupquote{descend\_changed}}}{}{}
\sphinxAtStartPar
Function for handling the descend checkbox being pressed. This will update the ascend setting and deselect the ascend checkbox.
:return: None
:rtype: None

\end{fulllineitems}

\index{duet\_port (app.MainWindow attribute)@\spxentry{duet\_port}\spxextra{app.MainWindow attribute}}

\begin{fulllineitems}
\phantomsection\label{\detokenize{index:app.MainWindow.duet_port}}\pysigline{\sphinxbfcode{\sphinxupquote{duet\_port}}\sphinxbfcode{\sphinxupquote{\DUrole{w}{  }\DUrole{p}{=}\DUrole{w}{  }\textquotesingle{}\textquotesingle{}}}}
\sphinxAtStartPar
The name of the port the duet is connected to

\end{fulllineitems}

\index{find\_symmetry\_axis() (app.MainWindow method)@\spxentry{find\_symmetry\_axis()}\spxextra{app.MainWindow method}}

\begin{fulllineitems}
\phantomsection\label{\detokenize{index:app.MainWindow.find_symmetry_axis}}\pysiglinewithargsret{\sphinxbfcode{\sphinxupquote{find\_symmetry\_axis}}}{\emph{\DUrole{n}{x}}, \emph{\DUrole{n}{y}}}{}
\sphinxAtStartPar
Function for calculating the point of symmetry of a a symmetric curve
:param x: List of x coordinates 
:param y: List of y coordinates
:return: The oint of symmetry
:rtype: float

\end{fulllineitems}

\index{func() (app.MainWindow method)@\spxentry{func()}\spxextra{app.MainWindow method}}

\begin{fulllineitems}
\phantomsection\label{\detokenize{index:app.MainWindow.func}}\pysiglinewithargsret{\sphinxbfcode{\sphinxupquote{func}}}{\emph{\DUrole{n}{x}}, \emph{\DUrole{n}{o}}, \emph{\DUrole{n}{a}}, \emph{\DUrole{n}{b}}, \emph{\DUrole{n}{c}}, \emph{\DUrole{n}{d}}, \emph{\DUrole{n}{e}}}{}
\sphinxAtStartPar
Polynomial function fitted to the measured inductance curve to determine the point of symmetry
:param x: List of x coordinates at which the function should be evaluated
:param o: The point of symmetry
:param a: Constant offset
:param b: Constant before the square
:param c: Constant before the to the power 4
:param d: Constant before the to the power 6
:param e: Constant before the to the power 8
:return: The output of the polynomial function
:rtype: Boolean

\end{fulllineitems}

\index{load\_settings() (app.MainWindow method)@\spxentry{load\_settings()}\spxextra{app.MainWindow method}}

\begin{fulllineitems}
\phantomsection\label{\detokenize{index:app.MainWindow.load_settings}}\pysiglinewithargsret{\sphinxbfcode{\sphinxupquote{load\_settings}}}{}{}
\sphinxAtStartPar
Function for loading settings to a settings.yaml file
:return: False if unsuccesful, True if succesfull
:rtype: Boolean

\end{fulllineitems}

\index{offset\_direction (app.MainWindow attribute)@\spxentry{offset\_direction}\spxextra{app.MainWindow attribute}}

\begin{fulllineitems}
\phantomsection\label{\detokenize{index:app.MainWindow.offset_direction}}\pysigline{\sphinxbfcode{\sphinxupquote{offset\_direction}}\sphinxbfcode{\sphinxupquote{\DUrole{w}{  }\DUrole{p}{=}\DUrole{w}{  }True}}}
\sphinxAtStartPar
If True the last run calibration was in the x direction, if False it was in the y direction

\end{fulllineitems}

\index{offset\_list (app.MainWindow attribute)@\spxentry{offset\_list}\spxextra{app.MainWindow attribute}}

\begin{fulllineitems}
\phantomsection\label{\detokenize{index:app.MainWindow.offset_list}}\pysigline{\sphinxbfcode{\sphinxupquote{offset\_list}}\sphinxbfcode{\sphinxupquote{\DUrole{w}{  }\DUrole{p}{=}\DUrole{w}{  }{[}{]}}}}
\sphinxAtStartPar
A list with the last found tool offsets belonging to the tools in {\hyperref[\detokenize{index:app.MainWindow.offset_tool_list}]{\sphinxcrossref{\sphinxcode{\sphinxupquote{MainWindow.offset\_tool\_list}}}}}

\end{fulllineitems}

\index{offset\_tool\_list (app.MainWindow attribute)@\spxentry{offset\_tool\_list}\spxextra{app.MainWindow attribute}}

\begin{fulllineitems}
\phantomsection\label{\detokenize{index:app.MainWindow.offset_tool_list}}\pysigline{\sphinxbfcode{\sphinxupquote{offset\_tool\_list}}\sphinxbfcode{\sphinxupquote{\DUrole{w}{  }\DUrole{p}{=}\DUrole{w}{  }{[}{]}}}}
\sphinxAtStartPar
A list of the tool numbers belonging to the tool offsets in {\hyperref[\detokenize{index:app.MainWindow.offset_list}]{\sphinxcrossref{\sphinxcode{\sphinxupquote{MainWindow.offset\_list}}}}}

\end{fulllineitems}

\index{output\_to\_terminal() (app.MainWindow method)@\spxentry{output\_to\_terminal()}\spxextra{app.MainWindow method}}

\begin{fulllineitems}
\phantomsection\label{\detokenize{index:app.MainWindow.output_to_terminal}}\pysiglinewithargsret{\sphinxbfcode{\sphinxupquote{output\_to\_terminal}}}{\emph{\DUrole{n}{new\_text}}}{}
\sphinxAtStartPar
Function for writing output to the terminal text box.
\begin{quote}\begin{description}
\item[{Returns}] \leavevmode
\sphinxAtStartPar
None

\item[{Return type}] \leavevmode
\sphinxAtStartPar
None

\end{description}\end{quote}

\end{fulllineitems}

\index{reload() (app.MainWindow method)@\spxentry{reload()}\spxextra{app.MainWindow method}}

\begin{fulllineitems}
\phantomsection\label{\detokenize{index:app.MainWindow.reload}}\pysiglinewithargsret{\sphinxbfcode{\sphinxupquote{reload}}}{}{}
\sphinxAtStartPar
Scans all COM ports and checks the name of all COM ports. If a name with “USB Serial Device” or “Duet” is found this it is selected as the printer port. If a name with ‘EVM’ is found this port is selected to be the port with the LDC1101EVM.
\begin{quote}\begin{description}
\item[{Returns}] \leavevmode
\sphinxAtStartPar
None

\item[{Return type}] \leavevmode
\sphinxAtStartPar
None

\end{description}\end{quote}

\end{fulllineitems}

\index{save\_settings() (app.MainWindow method)@\spxentry{save\_settings()}\spxextra{app.MainWindow method}}

\begin{fulllineitems}
\phantomsection\label{\detokenize{index:app.MainWindow.save_settings}}\pysiglinewithargsret{\sphinxbfcode{\sphinxupquote{save\_settings}}}{}{}
\sphinxAtStartPar
Function for saving settings to a settings.yaml file
:return: None
:rtype: None

\end{fulllineitems}

\index{stop() (app.MainWindow method)@\spxentry{stop()}\spxextra{app.MainWindow method}}

\begin{fulllineitems}
\phantomsection\label{\detokenize{index:app.MainWindow.stop}}\pysiglinewithargsret{\sphinxbfcode{\sphinxupquote{stop}}}{}{}
\sphinxAtStartPar
Function for handling the stop button being pressed. This will set a variable that will stop the running processes when possible.
\begin{quote}\begin{description}
\item[{Returns}] \leavevmode
\sphinxAtStartPar
None

\item[{Return type}] \leavevmode
\sphinxAtStartPar
None

\end{description}\end{quote}

\end{fulllineitems}

\index{stop\_button\_clicked (app.MainWindow attribute)@\spxentry{stop\_button\_clicked}\spxextra{app.MainWindow attribute}}

\begin{fulllineitems}
\phantomsection\label{\detokenize{index:app.MainWindow.stop_button_clicked}}\pysigline{\sphinxbfcode{\sphinxupquote{stop\_button\_clicked}}\sphinxbfcode{\sphinxupquote{\DUrole{w}{  }\DUrole{p}{=}\DUrole{w}{  }False}}}
\sphinxAtStartPar
Becomes True if the stop button has been clicked, until the measurement is stopped, then it becomes False again

\end{fulllineitems}

\index{test\_sensor() (app.MainWindow method)@\spxentry{test\_sensor()}\spxextra{app.MainWindow method}}

\begin{fulllineitems}
\phantomsection\label{\detokenize{index:app.MainWindow.test_sensor}}\pysiglinewithargsret{\sphinxbfcode{\sphinxupquote{test\_sensor}}}{}{}
\sphinxAtStartPar
Function for handling the test sensor checkbox being pressed. This will just record the LDC1101EVM sensor values until the stop button is clicked and store the result in the file specified in the filename textbox.
:return: None
:rtype: None

\end{fulllineitems}

\index{update\_tool\_list() (app.MainWindow method)@\spxentry{update\_tool\_list()}\spxextra{app.MainWindow method}}

\begin{fulllineitems}
\phantomsection\label{\detokenize{index:app.MainWindow.update_tool_list}}\pysiglinewithargsret{\sphinxbfcode{\sphinxupquote{update\_tool\_list}}}{}{}
\sphinxAtStartPar
Function for reading out the selected tools and the reference tool and putting them in the right order. The reference tool always will go first, then the other tools follow in either ascending or descending order, depending on whether ascend or descend is selected.
\begin{quote}\begin{description}
\item[{Returns}] \leavevmode
\sphinxAtStartPar
None

\item[{Return type}] \leavevmode
\sphinxAtStartPar
None

\end{description}\end{quote}

\end{fulllineitems}


\end{fulllineitems}

\index{main() (in module app)@\spxentry{main()}\spxextra{in module app}}

\begin{fulllineitems}
\phantomsection\label{\detokenize{index:app.main}}\pysiglinewithargsret{\sphinxcode{\sphinxupquote{app.}}\sphinxbfcode{\sphinxupquote{main}}}{}{}
\end{fulllineitems}



\chapter{diabase class}
\label{\detokenize{index:module-diabase}}\label{\detokenize{index:diabase-class}}\index{module@\spxentry{module}!diabase@\spxentry{diabase}}\index{diabase@\spxentry{diabase}!module@\spxentry{module}}\phantomsection\label{\detokenize{index:module-1}}\index{module@\spxentry{module}!diabase@\spxentry{diabase}}\index{diabase@\spxentry{diabase}!module@\spxentry{module}}\index{diabase (class in diabase)@\spxentry{diabase}\spxextra{class in diabase}}

\begin{fulllineitems}
\phantomsection\label{\detokenize{index:diabase.diabase}}\pysiglinewithargsret{\sphinxbfcode{\sphinxupquote{class\DUrole{w}{  }}}\sphinxcode{\sphinxupquote{diabase.}}\sphinxbfcode{\sphinxupquote{diabase}}}{\emph{\DUrole{n}{port}}}{}
\sphinxAtStartPar
Bases: \sphinxcode{\sphinxupquote{object}}
\index{attempts (diabase.diabase attribute)@\spxentry{attempts}\spxextra{diabase.diabase attribute}}

\begin{fulllineitems}
\phantomsection\label{\detokenize{index:diabase.diabase.attempts}}\pysigline{\sphinxbfcode{\sphinxupquote{attempts}}\sphinxbfcode{\sphinxupquote{\DUrole{w}{  }\DUrole{p}{=}\DUrole{w}{  }1000}}}
\sphinxAtStartPar
The number of lines to read before deciding the ‘OK’  from the printer will never arrive

\end{fulllineitems}

\index{close() (diabase.diabase method)@\spxentry{close()}\spxextra{diabase.diabase method}}

\begin{fulllineitems}
\phantomsection\label{\detokenize{index:diabase.diabase.close}}\pysiglinewithargsret{\sphinxbfcode{\sphinxupquote{close}}}{}{}
\sphinxAtStartPar
Function for closing the serial communication with the printer
\begin{quote}\begin{description}
\item[{Returns}] \leavevmode
\sphinxAtStartPar
None

\item[{Return type}] \leavevmode
\sphinxAtStartPar
None

\end{description}\end{quote}

\end{fulllineitems}

\index{get\_current\_position() (diabase.diabase method)@\spxentry{get\_current\_position()}\spxextra{diabase.diabase method}}

\begin{fulllineitems}
\phantomsection\label{\detokenize{index:diabase.diabase.get_current_position}}\pysiglinewithargsret{\sphinxbfcode{\sphinxupquote{get\_current\_position}}}{}{}
\sphinxAtStartPar
Function for getting the current position of the printer using a M114 command
\begin{quote}\begin{description}
\item[{Returns}] \leavevmode
\sphinxAtStartPar
Dict with the current position. The dict contains a key ‘x’, ‘y’ or ‘z’ with the current position in the corresponding direction.

\item[{Return type}] \leavevmode
\sphinxAtStartPar
Dict

\end{description}\end{quote}

\end{fulllineitems}

\index{set\_tool\_offset() (diabase.diabase method)@\spxentry{set\_tool\_offset()}\spxextra{diabase.diabase method}}

\begin{fulllineitems}
\phantomsection\label{\detokenize{index:diabase.diabase.set_tool_offset}}\pysiglinewithargsret{\sphinxbfcode{\sphinxupquote{set\_tool\_offset}}}{\emph{\DUrole{n}{tool}}, \emph{\DUrole{n}{pos}}}{}
\sphinxAtStartPar
Function for setting tool offsets.
\begin{quote}\begin{description}
\item[{Parameters}] \leavevmode\begin{itemize}
\item {} 
\sphinxAtStartPar
\sphinxstyleliteralstrong{\sphinxupquote{tool}} \textendash{} The tool number of the tool of which to set the offsets

\item {} 
\sphinxAtStartPar
\sphinxstyleliteralstrong{\sphinxupquote{pos}} \textendash{} Dict with the tool offsets. The function expect a key ‘x’, ‘y’ or ‘z’ with the tool offset in the corresponding direction.

\end{itemize}

\item[{Returns}] \leavevmode
\sphinxAtStartPar
None

\item[{Return type}] \leavevmode
\sphinxAtStartPar
None

\end{description}\end{quote}

\end{fulllineitems}

\index{set\_tool\_offset\_differential() (diabase.diabase method)@\spxentry{set\_tool\_offset\_differential()}\spxextra{diabase.diabase method}}

\begin{fulllineitems}
\phantomsection\label{\detokenize{index:diabase.diabase.set_tool_offset_differential}}\pysiglinewithargsret{\sphinxbfcode{\sphinxupquote{set\_tool\_offset\_differential}}}{\emph{\DUrole{n}{tool}}, \emph{\DUrole{n}{extra\_offset}}}{}
\sphinxAtStartPar
Function for setting tool offsets relative to the current tool offsets. To do so the printer will:
\begin{itemize}
\item {} 
\sphinxAtStartPar
Select the tool

\item {} 
\sphinxAtStartPar
Get the current position

\item {} 
\sphinxAtStartPar
Set the tool offset to zero

\item {} 
\sphinxAtStartPar
Measure the position again

\item {} 
\sphinxAtStartPar
\sphinxhyphen{}Set the tool offset to the last measured tool offset plus the addional tool offset

\end{itemize}
\begin{quote}\begin{description}
\item[{Parameters}] \leavevmode\begin{itemize}
\item {} 
\sphinxAtStartPar
\sphinxstyleliteralstrong{\sphinxupquote{tool}} \textendash{} The tool number of the tool of which to set the offsets

\item {} 
\sphinxAtStartPar
\sphinxstyleliteralstrong{\sphinxupquote{extra\_offset}} \textendash{} Dict with the additional tool offsets. The function expect a key ‘x’, ‘y’ or ‘z’ with the additional tool offset in the corresponding direction.

\end{itemize}

\item[{Returns}] \leavevmode
\sphinxAtStartPar
None

\item[{Return type}] \leavevmode
\sphinxAtStartPar
None

\end{description}\end{quote}

\end{fulllineitems}

\index{store\_offset\_parameters() (diabase.diabase method)@\spxentry{store\_offset\_parameters()}\spxextra{diabase.diabase method}}

\begin{fulllineitems}
\phantomsection\label{\detokenize{index:diabase.diabase.store_offset_parameters}}\pysiglinewithargsret{\sphinxbfcode{\sphinxupquote{store\_offset\_parameters}}}{}{}
\sphinxAtStartPar
Function for storing the current tool offsets in flash such that they will still be there when the printer is restarted.
\begin{quote}\begin{description}
\item[{Returns}] \leavevmode
\sphinxAtStartPar
None

\item[{Return type}] \leavevmode
\sphinxAtStartPar
None

\end{description}\end{quote}

\end{fulllineitems}

\index{write\_line() (diabase.diabase method)@\spxentry{write\_line()}\spxextra{diabase.diabase method}}

\begin{fulllineitems}
\phantomsection\label{\detokenize{index:diabase.diabase.write_line}}\pysiglinewithargsret{\sphinxbfcode{\sphinxupquote{write\_line}}}{\emph{\DUrole{n}{string}}}{}
\sphinxAtStartPar
Write a line of GCODE to the printer. This function will wait for an ‘OK’ from the printer, meaning that the command has finished executing (except for G1 commands). If it takes too to many attempts for the printer give an answer it will be assumed something went wrong and the function will return anyways.
\begin{quote}\begin{description}
\item[{Parameters}] \leavevmode
\sphinxAtStartPar
\sphinxstyleliteralstrong{\sphinxupquote{string}} \textendash{} The line of GCODE to write to the printer.

\item[{Returns}] \leavevmode
\sphinxAtStartPar
None

\item[{Return type}] \leavevmode
\sphinxAtStartPar
None

\end{description}\end{quote}

\end{fulllineitems}


\end{fulllineitems}



\chapter{ldc1101evm class}
\label{\detokenize{index:ldc1101evm-class}}\phantomsection\label{\detokenize{index:module-ldc1101evm}}\index{module@\spxentry{module}!ldc1101evm@\spxentry{ldc1101evm}}\index{ldc1101evm@\spxentry{ldc1101evm}!module@\spxentry{module}}\phantomsection\label{\detokenize{index:module-2}}\index{module@\spxentry{module}!ldc1101evm@\spxentry{ldc1101evm}}\index{ldc1101evm@\spxentry{ldc1101evm}!module@\spxentry{module}}\index{ldc1101evm (class in ldc1101evm)@\spxentry{ldc1101evm}\spxextra{class in ldc1101evm}}

\begin{fulllineitems}
\phantomsection\label{\detokenize{index:ldc1101evm.ldc1101evm}}\pysiglinewithargsret{\sphinxbfcode{\sphinxupquote{class\DUrole{w}{  }}}\sphinxcode{\sphinxupquote{ldc1101evm.}}\sphinxbfcode{\sphinxupquote{ldc1101evm}}}{\emph{\DUrole{n}{port}}}{}
\sphinxAtStartPar
Bases: \sphinxcode{\sphinxupquote{object}}
\index{Csensor (ldc1101evm.ldc1101evm attribute)@\spxentry{Csensor}\spxextra{ldc1101evm.ldc1101evm attribute}}

\begin{fulllineitems}
\phantomsection\label{\detokenize{index:ldc1101evm.ldc1101evm.Csensor}}\pysigline{\sphinxbfcode{\sphinxupquote{Csensor}}\sphinxbfcode{\sphinxupquote{\DUrole{w}{  }\DUrole{p}{=}\DUrole{w}{  }1.2e\sphinxhyphen{}09}}}
\sphinxAtStartPar
Value of the capacitor soldered onto the LDC1101EVM. This will affect the measured inductance since the LDC1101EVM determines the osciallation frequency of an LC tank with this capaictor and the inductor to be measured.

\end{fulllineitems}

\index{LHR\_init() (ldc1101evm.ldc1101evm method)@\spxentry{LHR\_init()}\spxextra{ldc1101evm.ldc1101evm method}}

\begin{fulllineitems}
\phantomsection\label{\detokenize{index:ldc1101evm.ldc1101evm.LHR_init}}\pysiglinewithargsret{\sphinxbfcode{\sphinxupquote{LHR\_init}}}{}{}
\sphinxAtStartPar
Function for initialising a high resolution measurement. A high resolution measurement is 24 bit and has no R measurement.
\begin{quote}\begin{description}
\item[{Returns}] \leavevmode
\sphinxAtStartPar
None

\item[{Return type}] \leavevmode
\sphinxAtStartPar
None

\end{description}\end{quote}

\end{fulllineitems}

\index{close() (ldc1101evm.ldc1101evm method)@\spxentry{close()}\spxextra{ldc1101evm.ldc1101evm method}}

\begin{fulllineitems}
\phantomsection\label{\detokenize{index:ldc1101evm.ldc1101evm.close}}\pysiglinewithargsret{\sphinxbfcode{\sphinxupquote{close}}}{}{}
\sphinxAtStartPar
Close the serial connection and tell the daemon to go kill itself.
\begin{quote}\begin{description}
\item[{Returns}] \leavevmode
\sphinxAtStartPar
None

\item[{Return type}] \leavevmode
\sphinxAtStartPar
None

\end{description}\end{quote}

\end{fulllineitems}

\index{flush() (ldc1101evm.ldc1101evm method)@\spxentry{flush()}\spxextra{ldc1101evm.ldc1101evm method}}

\begin{fulllineitems}
\phantomsection\label{\detokenize{index:ldc1101evm.ldc1101evm.flush}}\pysiglinewithargsret{\sphinxbfcode{\sphinxupquote{flush}}}{}{}
\sphinxAtStartPar
Delete all currently stored measurements
\begin{quote}\begin{description}
\item[{Returns}] \leavevmode
\sphinxAtStartPar
None

\item[{Return type}] \leavevmode
\sphinxAtStartPar
None

\end{description}\end{quote}

\end{fulllineitems}

\index{get\_LHR\_data() (ldc1101evm.ldc1101evm method)@\spxentry{get\_LHR\_data()}\spxextra{ldc1101evm.ldc1101evm method}}

\begin{fulllineitems}
\phantomsection\label{\detokenize{index:ldc1101evm.ldc1101evm.get_LHR_data}}\pysiglinewithargsret{\sphinxbfcode{\sphinxupquote{get\_LHR\_data}}}{\emph{\DUrole{n}{down\_sample\_ratio}}}{}
\sphinxAtStartPar
Function getting the inductance measured by the LDC1101EVM in LHR mode. To put it in LHR mode run {\hyperref[\detokenize{index:ldc1101evm.ldc1101evm.LHR_init}]{\sphinxcrossref{\sphinxcode{\sphinxupquote{ldc1101evm.LHR\_init()}}}}} first. This function blocks until an inductance value that has not been read is available. To delete all currently stored measurements run {\hyperref[\detokenize{index:ldc1101evm.ldc1101evm.flush}]{\sphinxcrossref{\sphinxcode{\sphinxupquote{ldc1101evm.flush()}}}}} first.
\begin{quote}\begin{description}
\item[{Parameters}] \leavevmode
\sphinxAtStartPar
\sphinxstyleliteralstrong{\sphinxupquote{down\_sample\_ratio}} \textendash{} How much the output should be downsampled. This reduces the sampling rate but increases the effective resolution by taking the average.

\item[{Returns}] \leavevmode
\sphinxAtStartPar
The measured inductance

\item[{Return type}] \leavevmode
\sphinxAtStartPar
float

\end{description}\end{quote}

\end{fulllineitems}

\index{lock (ldc1101evm.ldc1101evm attribute)@\spxentry{lock}\spxextra{ldc1101evm.ldc1101evm attribute}}

\begin{fulllineitems}
\phantomsection\label{\detokenize{index:ldc1101evm.ldc1101evm.lock}}\pysigline{\sphinxbfcode{\sphinxupquote{lock}}\sphinxbfcode{\sphinxupquote{\DUrole{w}{  }\DUrole{p}{=}\DUrole{w}{  }\textless{}unlocked \_thread.lock object\textgreater{}}}}
\sphinxAtStartPar
Mutex for making sure the serial daemon and the other functions don’t try to access {\hyperref[\detokenize{index:ldc1101evm.ldc1101evm.received_bytes}]{\sphinxcrossref{\sphinxcode{\sphinxupquote{ldc1101evm.received\_bytes}}}}} at the same time

\end{fulllineitems}

\index{received\_bytes (ldc1101evm.ldc1101evm attribute)@\spxentry{received\_bytes}\spxextra{ldc1101evm.ldc1101evm attribute}}

\begin{fulllineitems}
\phantomsection\label{\detokenize{index:ldc1101evm.ldc1101evm.received_bytes}}\pysigline{\sphinxbfcode{\sphinxupquote{received\_bytes}}\sphinxbfcode{\sphinxupquote{\DUrole{w}{  }\DUrole{p}{=}\DUrole{w}{  }b\textquotesingle{}\textquotesingle{}}}}
\sphinxAtStartPar
Stores all the bytes received from the LDC1101EVM

\end{fulllineitems}

\index{serial\_daemon() (ldc1101evm.ldc1101evm method)@\spxentry{serial\_daemon()}\spxextra{ldc1101evm.ldc1101evm method}}

\begin{fulllineitems}
\phantomsection\label{\detokenize{index:ldc1101evm.ldc1101evm.serial_daemon}}\pysiglinewithargsret{\sphinxbfcode{\sphinxupquote{serial\_daemon}}}{}{}
\sphinxAtStartPar
The serial daemon which is run in a seperate thread as the rest and just puts all the received bytes in {\hyperref[\detokenize{index:ldc1101evm.ldc1101evm.received_bytes}]{\sphinxcrossref{\sphinxcode{\sphinxupquote{ldc1101evm.received\_bytes}}}}}
\begin{quote}\begin{description}
\item[{Returns}] \leavevmode
\sphinxAtStartPar
None

\item[{Return type}] \leavevmode
\sphinxAtStartPar
None

\end{description}\end{quote}

\end{fulllineitems}

\index{stop\_thread (ldc1101evm.ldc1101evm attribute)@\spxentry{stop\_thread}\spxextra{ldc1101evm.ldc1101evm attribute}}

\begin{fulllineitems}
\phantomsection\label{\detokenize{index:ldc1101evm.ldc1101evm.stop_thread}}\pysigline{\sphinxbfcode{\sphinxupquote{stop\_thread}}\sphinxbfcode{\sphinxupquote{\DUrole{w}{  }\DUrole{p}{=}\DUrole{w}{  }False}}}
\sphinxAtStartPar
If set to True, the serial daemon will kill itself

\end{fulllineitems}


\end{fulllineitems}



%\chapter{Indices and tables}
%\label{\detokenize{index:indices-and-tables}}\begin{itemize}
%\item {} 
%\sphinxAtStartPar
%\DUrole{xref,std,std-ref}{genindex}
%
%\item {} 
%\sphinxAtStartPar
%\DUrole{xref,std,std-ref}{modindex}
%
%\item {} 
%\sphinxAtStartPar
%\DUrole{xref,std,std-ref}{search}
%
%\end{itemize}


\renewcommand{\indexname}{Python Module Index}
\begin{sphinxtheindex}
\let\bigletter\sphinxstyleindexlettergroup
\bigletter{a}
\item\relax\sphinxstyleindexentry{app}\sphinxstyleindexpageref{index:\detokenize{module-0}}
\indexspace
\bigletter{d}
\item\relax\sphinxstyleindexentry{diabase}\sphinxstyleindexpageref{index:\detokenize{module-1}}
\indexspace
\bigletter{l}
\item\relax\sphinxstyleindexentry{ldc1101evm}\sphinxstyleindexpageref{index:\detokenize{module-2}}
\end{sphinxtheindex}

\renewcommand{\indexname}{Index}
\printindex
\end{document}